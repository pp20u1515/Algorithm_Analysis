\chapter*{\hfill{\centering ЗАКЛЮЧЕНИЕ}\hfill}
\addcontentsline{toc}{chapter}{ЗАКЛЮЧЕНИЕ}

В результате исследования параллельного и последовательного алгоритмов исправления ошибок в тексте были использованы мьютексы для обеспечения корректного взаимодействия между потоками. 
Оказалось, что при увеличении числа потоков до 4 потока, производительность параллельной реализации значительно превосходит последовательную. 
Однако, при дальнейшем увеличении числа потоков до 8 наблюдается некоторое снижение эффективности из - за увеличения накладных расходов на управление потоками. 
Несмотря на это, алгоритм с 8 потоками остается более быстрым, чем последовательная реализация.

При использовании 16 и более потоков параллельная реализация проигрывает последовательной. 
Это связано с появлением проблемы конкуренции за ресурсы процессора, созданием большого числа потоков и увеличением объема обслуживаемой очереди, что в итоге снижает общую производительность.

В результате выполнения лабораторной работы были выполнены следующие задачи:

\begin{enumerate}[label={\arabic*)}]
	\item проанализированы параллельный и последовательный варианты алгоритма исправления орфографических ошибок в тексте;
	\item определены средства программной реализации выбранного алгоритма;
	\item реализованы параллельную и последовательную версии алгоритма исправления орфографических ошибок в тексте;
	\item проведен сравнительный анализ по времени реализованного алгоритма;
	\item подготовлен отчет о лабораторной работе.
\end{enumerate}