\chapter*{\hfill{\centering ВВЕДЕНИЕ}\hfill}
\addcontentsline{toc}{chapter}{ВВЕДЕНИЕ}

С появлением новых вычислительных систем программисты стали сталкиваться с необходимостью проведения одновременной обработки данных для улучшения отзывчивости системы, ускорения выполнения вычислений и более эффективного использования вычислительных ресурсов.
Развитие процессоров позволило использовать один процессор для выполнения нескольких параллельных операций, что привело к появлению термина «Многопоточность».

Целью данной лабораторной работы является получение навыков организации параллельного выполнения операций.

Для достижения цели, требуется выполнить следующие задачи:
\begin{enumerate}[label={\arabic*)}]
	\item проанализировать последовательный и параллельный варианты алгоритма исправления орфографических ошибок в тексте;
	\item определить средства программной реализации выбранного алгоритма;
	\item реализовать параллельную и последовательную версии алгоритма исправления орфографических ошибок в тексте;
	\item провести сравнительный анализ по времени реализованного алгоритма;
	\item подготовить отчет о лабораторной работе.
\end{enumerate}