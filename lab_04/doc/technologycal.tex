\chapter{Технологическая часть}

В этом разделе предоставляются листинги реализованных алгоритмов и
осуществляется выбор средств реализации.

\section{Выбор средств реализации}

Для выполнения данной лабораторной работы был выбран язык программирования C++. 
Время измерялось с помощью функции clock() из библиотеки time.h~\cite{clock_t}.
Для реализации синхронизации в программе были использованы мютексы, обеспечивающие безопасный доступ к общим ресурсам~\cite{mutex}.
Их применение позволяет избежать гонок данных и обеспечить корректное выполнение потоков.


\section{Реализация алгоритмов}

В листинге 3.1 представлена реализация алгоритма поиска расстояния Левенштейна.

\clearpage

\begin{lstlisting}[caption=Алгоритм поиска расстояния Левенштейна]
	int alg_lev(const string &word1, const string &word2)
	{
		const size_t n = word1.length();
		const size_t m = word2.length();
		int change = 0; 
		vector<vector<int>> matrix(n + 1, vector<int>(m + 1, 0));
		
		init_matrix(matrix, n + 1, m + 1);
		
		for (size_t index_i = 1; index_i < n + 1; index_i++)
		for (size_t index_j = 1; index_j < m + 1; index_j++)
		{
			change = word1[index_i - 1] == word2[index_j - 1] ? EQUAL : NOT_EQUAL;
			matrix[index_i][index_j] = min(matrix[index_i][index_j - 1] + 1,
			min(matrix[index_i - 1][index_j] + 1,
			matrix[index_i - 1][index_j - 1] + change));
		}
		return matrix[n][m];
	}
\end{lstlisting}

\clearpage

В листинге 3.2 представлена реализация последовательного алгоритма исправления орфографических ошибок в тексте.

\begin{lstlisting}[caption=Последовательный алгоритм исправления орфографических ошибок в тексте]
	void single_thread(const string &input_text, const vector<string> &dictionary)
	{
		string temp_text = input_text;
		istringstream iss(temp_text); 
		vector<string> words(istream_iterator<string>{iss}, istream_iterator<string>{}); 
		for (auto &word : words) 
		{
			int min_distance = numeric_limits<int>::max(); 
			string best_match = word; 
			
			for (auto &dict_word : dictionary) 
			{
				int distance = alg_lev(word, dict_word); 
				
				if (distance < min_distance) 
				{
					min_distance = distance; 
					best_match = dict_word; 
				}
			}
			size_t pos = temp_text.find(word);
			
			while (pos != string::npos) 
			{
				temp_text.replace(pos, word.length(), best_match); 
				pos = temp_text.find(word, pos + best_match.length()); 
			}
		}
	}
\end{lstlisting}

\clearpage

В листинге 3.3 представлена реализация многопоточного алгоритма исправления орфографических ошибок в тексте.

\begin{lstlisting}[caption=Многопоточный алгоритм исправления орфографических ошибок в тексте]
	void multi_thread(const string &input_text, const vector<string> &dictionary, int num_threads) 
	{
		vector<string> words;
		istringstream iss(input_text);
		copy(istream_iterator<string>(iss), istream_iterator<string>(), back_inserter(words));
		
		vector<thread> threads;
		vector<string> correctedWords;
		mutex resultMutex;
		
		int wordsPerThread = words.size() / num_threads;
		int remainder = words.size() % num_threads;
		
		for (int i = 0; i < num_threads; ++i) {
			int start = i * wordsPerThread;
			int end = start + wordsPerThread + (i == num_threads - 1 ? remainder : 0);
			
			threads.emplace_back(processWords, words.begin() + start, words.begin() + end,
			ref(dictionary), ref(correctedWords), ref(resultMutex));
		}
		
		for (auto& thread : threads) {
			thread.join();
		}
	}
\end{lstlisting}

\section*{Вывод}

Были выбраны инструменты для реализации и разработаны последовательный и многопоточный алгориты исправления орфографических ошибок в тексте.
Также предоставлены листинги кода на выбранном языке программирования.