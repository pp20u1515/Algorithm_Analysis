\chapter{Аналитическая часть}

Расстояния Левенштейна и Дамерау - Левенштейна (редакционное расстояние, дистанция редактирования) -- метрика, измеряющая по модулю разность между двумя последовательностями символов. 
Она определяется как минимальное количество односимвольных операций (а именно вставки -- I, удаления -- D, замены -- R), необходимых для превращения одной последовательности символов в другую. 
Также вводится операция, которая не требует никаких действий -- совпадение -- M.

В расстоянии Дамерау - Левенштейна, помимо вышеупомянутых операций, также предусматривается возможность перестановки соседних символов, что обозначается как операция X.

Kaждой из этих операций можно приписать определенный штраф.
Обы-чно используется сделующий набор штрафов: для операции М (совпадение) штраф равен нулю, а для операций I (вставка), D (удаление), R (замена), X (перестановка) штраф составляет единицу.

Таким образом, задача по вычислению расстояния Дамерау - Левенштейна сводится к нахождению последовательности операций, которая минимизирует общую сумму штрафов. 
Эту задачу можно решить с помощью использования рекуррентных формул.

\section{Расстояние Левенштейна}

Пусть дано две строки $S_{1}$ и $S_{2}$. Тогда расстояние Левенштейна можно найти по рекуррентной формуле (\ref{eq:ref1}):

\begin{equation}
	\label{eq:ref1}
	D(i, j) =
	\begin{cases}
		0, &\text{i = 0, j = 0}\\
		i, &\text{j = 0, i > 0}\\
		j, &\text{i = 0, j > 0}\\
		min \begin{cases}
			D(i, j - 1) + 1,\\
			D(i - 1, j) + 1,\\
			D(i - 1, j - 1) +  m(S_{1}[i], S_{2}[j]), \\
		\end{cases}
		&\text{i > 0, j > 0}
	\end{cases}
\end{equation}

Первые три формулы в системе (1.1) можно рассматривать как базовые и они представляют следующее: первое уравнение описывает ситуацию, когда не требуется никаких действий (символы обеих строк совпадают, так как обе строки пусты).
Во втором уравнении рассматривается вставка ј символов в пустую строку $S_{1}$ для создания строки-копии $S_{2}$ длиной j. 
В третьем уравнении описывается удаление всех i символов из строки $S_{1}$ для того, чтобы она соответствовала пустой строке $S_{2}$. 

Последующие шаги в алгоритме требуют выбора минимального значения из штрафов, которые могут возникнуть в результате следующих операций: вставки символа в $S_{1}$ (первое уравнение в группе min), удаление символа из $S_{1}$ (второе уравнение в группе min), а также совпадения или замены, в зависимости од того, совпадают ли рассматриваемые символы в данной точке строк (третье уравнение в группе min).


\section{Расстояние Дамерау - Левенштейна}

Для вычисления расстояния Дамерау - Левенштейна между строками $S_{1}$ и $S_{2}$ используется аналогичная реккурентная формула, как в~(\ref{eq:ref1}). 
Главное отличие заключается в том, что добавляем четвертую возможную операцию (\ref{eq:ref2}) в группу min:

\begin{equation}
	\left[ 
	\begin{array}{c} 
		D(S_1[1...i-2],S_2[1...j-2]) + 1, $ если $ $i, j>1, $a_i=b_{j-1}, b_j=a_{i-1}\\
		\infty $ , иначе$ 
	\end{array}
	\right.\\
	\label{eq:ref2}
\end{equation}

Этот сценарий предусматривает перестановку символов, расположенных рядом в строке $S_{1}$, только если длины обеих строк больше одного символа, и символы, которые рассматриваем, крест-накрест в строках $S_{1}$ и $S_{2}$, совпадают. 
В случае, если хотя бы одно из этих условий не выполняется, данное действие не учитывается при определении минимального расстояния.
	
Итоговая формула для расчета расстояния Дамерау - Левенштейна выглядит следующим образом~(\ref{eq:ref3}):

\begin{equation}
	\label{eq:ref3}
	D(i, j) = 
	\begin{cases}
		0, &\text{i = 0, j = 0,}\\
		i, &\text{j = 0, i > 0,}\\
		j, &\text{i = 0, j > 0,}\\
		\min (  D(i, j - 1) + 1,\\
		\qquad D(i - 1, j) + 1,\\
		\qquad D(i - 1, j - 1) + m(S_{1}[i], S_{2}[j]), \\
		\qquad D(i - 2, j - 2) + 1 ),
		& \begin{aligned}
			& \text{если i > 1, j > 1}, \\
			& S_{1}[i] = S_{2}[j - 1], \\
			& S_{1}[i - 1] = S_{2}[j], \\
		\end{aligned}\\
		
		\min ( D(i, j - 1) + 1,\\
		\qquad  D(i - 1, j) + 1, \\
		\qquad  D(i - 1, j - 1) + m(S_{1}[i], S_{2}[j]) ) & \text{, иначе.}
	\end{cases}
\end{equation}

\vspace{30mm}

\section*{Вывод}
В данной секции были изучены алгоритмы вычисления расстояния Левенштейна и его модификации -- расстояния Дамерау - Левенштейна, которое включает в себя возможность перестановки соседних символов. 
Формулы для расчета расстояния Левенштейна и Дамерау - Левенштейна между строками задаются через рекурсивные уравнения, что означает, что соответствующие алгоритмы могут быть реализованы как с использованием рекурсии, так и в итеративной форме.