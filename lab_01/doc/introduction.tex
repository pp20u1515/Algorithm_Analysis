\chapter*{Введение}
\addcontentsline{toc}{chapter}{Введение}

Расстояние Левенштейна, также известное как редакционное расстояние, может быть определено как последовательность операций, необходимых для привращения одной строки в другую с минимальным количеством шагов~\cite{levenshtein}. 

В расстоянии Левенштейна используются такие операции как вставка, удаление, замена символов, необходимых для преобразования одной строки в другую. 

Расстояние Дамерау - Левенштейна также представляет собой минимальное количество редакционных операций, которое дополнительно включает в себя операцию перестановки двух соседних символов~\cite{ulianov} .

Расстояния Левенштейна и Дамерау - Левенштейна находят применение в различных областях, таких как: 
\begin{itemize}
	\item компьютерная лингвистика (автозамена в посиковых запросах, текстовые редакторы);
	\item биоинформатика (анализ последовательностей белков);
	\item нечеткий поиск записей в базах (борьба с опечатками).
\end{itemize}

Целью данной лабораторной работы является изучение алгоритмов поиска редакционных расстояний Левенштейна и Дамерау - Левенштейна. 
Для успешного выполнения лабораторной работы необходимо выполнить следующие задачи: 
\begin{enumerate}[label={\arabic*)}]
	\item изучить методы вычисления расстояний Левенштейна и Дамерау - Левенштейна;
	\item разработать алгоритмы для вычисления указанных расстояний;
	\item реализовать разработанные алгоритмы в виде программного кода;
	\item провести анализ затрат реализаций алгоритмов по времени и по памяти;
	\item подготовить отчет по лабораторной работе.
\end{enumerate}