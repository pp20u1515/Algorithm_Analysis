\chapter{Технологическая часть}

В этом разделе предоставляются реализации выполненных алгоритмов и осуществляется выбор средств реализации.

\section{Выбор средств реализации}

Для выполнения данной лабораторной работы был выбран язык программирования C++.  
Время измерялось с помощью функции \textit{clock()} из библиотеки \textit{time.h}~\cite{time}.

\section{Реализация алгоритмов}

В листингах 3.1 -- 3.4 представлены реализации рассматриваемых алгоритмов.


\hspace{0.6cm}В Листинге 3.1 показана реализация нерекурсивного алгоритма нахождения расстояния Левенштейна.

\bigskip

\begin{lstlisting}[caption=Нерекурсивный алгоритм поиска расстояния Левенштейна]
	int normalLevenAlgorithm(const std::wstring &firstWord, const std::wstring &secondWord, bool key)
	{
		std::size_t firstLength = firstWord.length(); 
		std::size_t secondLength = secondWord.length(); 
		int answer = 0; 
		int change = 0; 
		
		std::vector<std::vector<int>> matrix(firstLength + 1, std::vector<int>(secondLength + 1, 0));
		
		initMatrix(matrix, firstLength + 1, secondLength + 1);
		
		for (std::size_t indexI = 1; indexI < firstLength + 1; indexI++)
			for (std::size_t indexJ = 1; indexJ < secondLength + 1; indexJ++)
			{
				change = checkEquality(firstWord[indexI - 1], secondWord[indexJ - 1]);
			
				matrix[indexI][indexJ] = std::min(matrix[indexI][indexJ - 1] + 1,
				std::min(matrix[indexI - 1][indexJ] + 1,
				matrix[indexI - 1][indexJ - 1] + change));
			}
		
		if (key)
			printMatrix(matrix, firstLength + 1, secondLength + 1);
		
		answer = matrix[firstLength][secondLength];     
		
		return answer;
	}
\end{lstlisting}

\newpage

\hspace{0.6cm}В Листинге 3.2 показана реализация нерекурсивного алгоритма нахождения расстояния Дамерау - Левенштейна.
\begin{lstlisting}[caption=Нерекурсивный алгоритм поиска расстояния Дамерау - Левенштейна]
	int normalLevenAlgorithm(const std::wstring &firstWord, const std::wstring &secondWord, bool key)
	{
		std::size_t firstLength = firstWord.length(); 
		std::size_t secondLength = secondWord.length();
		int answer = 0; 
		int change = 0; 
		
		std::vector<std::vector<int>> matrix(firstLength + 1, std::vector<int>(secondLength + 1, 0));
		
		initMatrix(matrix, firstLength + 1, secondLength + 1);
		
		for (std::size_t indexI = 1; indexI < firstLength + 1; indexI++)
			for (std::size_t indexJ = 1; indexJ < secondLength + 1; indexJ++)
			{
				change = checkEquality(firstWord[indexI - 1], secondWord[indexJ - 1]);
			
				matrix[indexI][indexJ] = std::min(matrix[indexI][indexJ - 1] + 1,
				std::min(matrix[indexI - 1][indexJ] + 1,
				matrix[indexI - 1][indexJ - 1] + change));
			}
		
		if (key)
			printMatrix(matrix, firstLength + 1, secondLength + 1);
		
		answer = matrix[firstLength][secondLength];     
		
		return answer;
	}
\end{lstlisting}

\vspace{60mm} 

В Листинге 3.3 показана реализация рекурсивного алгоритма нахождения расстояния Дамерау - Левенштейна без кэша.
\begin{lstlisting}[caption=Рекурсивный алгоритм поиска расстояния Дамерау - Левенштейна без кэша]
int recursiveDamerauLeven(const std::wstring firstWord, const std::wstring secondWord,
std::size_t firstLenght, std::size_t secondLength)
{
	int rc = 0;
	
	if (firstLenght != 0)
	{
		if (secondLength != 0)
		{
			int change = checkEquality(firstWord[firstLenght - 1], secondWord[secondLength - 1]);
			
			rc = std::min(recursiveDamerauLeven(firstWord, secondWord, firstLenght, secondLength - 1) + 1,
			std::min(recursiveDamerauLeven(firstWord, secondWord, firstLenght - 1, secondLength) + 1,
			recursiveDamerauLeven(firstWord, secondWord, firstLenght - 1, secondLength - 1) + change));
			
			if (firstLenght > 1 && secondLength > 1 && firstWord[firstLenght - 1] == secondWord[secondLength - 2] &&
			firstWord[firstLenght - 2] == secondWord[secondLength - 1])
				rc = std::min(rc, recursiveDamerauLeven(firstWord, secondWord, firstLenght - 2, secondLength - 2) + 1);
		}
		else
			rc = firstLenght;
	}
	else 
		rc = secondLength;    
	return rc;
	}
\end{lstlisting}

\vspace{50mm}

В Листинге 3.4 показана реализация рекурсивного алгоритма нахождения расстояния Дамерау - Левенштейна с кэшом.
\begin{lstlisting}[caption=Рекурсивный алгоритм поиска расстояния Дамерау - Левенштейна с кэшом]
	int cacheDamerauLeven(std::vector<std::vector<int>> &matrix, const std::wstring firstWord, const std::wstring secondWord,
	const std::size_t &firstLenght, const std::size_t &secondLength)
	{
		if (firstLenght != 0)
		{
			if (secondLength != 0)
			{
				int change = checkEquality(firstWord[firstLenght - 1], secondWord[secondLength - 1]);
				
				matrix[firstLenght][secondLength] = std::min(cacheDamerauLeven(matrix, 
				firstWord, secondWord, firstLenght, secondLength - 1) + 1,
				std::min(cacheDamerauLeven(matrix, firstWord, secondWord, firstLenght - 1, secondLength) + 1,
				cacheDamerauLeven(matrix, firstWord, secondWord, firstLenght - 1, secondLength - 1) + change));
				
				if (firstLenght > 1 && secondLength > 1 && firstWord[firstLenght - 1] == secondWord[secondLength - 2] &&
				firstWord[firstLenght - 2] == secondWord[secondLength - 1])
					matrix[firstLenght][secondLength] = std::min(matrix[firstLenght][secondLength], 
					cacheDamerauLeven(matrix, firstWord, secondWord, firstLenght - 2, secondLength - 2) + 1);
			}
			else
				matrix[firstLenght][secondLength] = firstLenght;
		}
		else
			matrix[firstLenght][secondLength] = secondLength;
		
		return matrix[firstLenght][secondLength];
	}
	}
\end{lstlisting}

\section{Тестовые данные}

В таблице 3.1 приведены входные данные, на которых было протестированно разработанное ПО.

\begin{table}[ht]
	\small
	\begin{center}
		\begin{threeparttable}
			\caption{Функциональные тесты}
			\label{tbl:func_tests}
			\begin{tabular}{|c|c|c|c|c|c|}
				\hline
				\multicolumn{2}{|c|}{\bfseries Входные данные}
				& \multicolumn{4}{c|}{\bfseries Расстояние и алгоритм} \\ 
				\hline 
				&
				& \multicolumn{1}{c|}{\bfseries Левенштейна} 
				& \multicolumn{3}{c|}{\bfseries Дамерау - Левенштейна} \\ \cline{3-6}
				
				\bfseries Строка 1 & \bfseries Строка 2 & \bfseries Итеративный & \bfseries Итеративный
				
				& \multicolumn{2}{c|}{\bfseries Рекурсивный} \\ \cline{5-6}
				& & & & \bfseries Без кеша & \bfseries С кешом \\
				\hline
				a & b & 1 & 1 & 1 & 1 \\
				\hline
				бабушка & бабушка & 0 & 0 & 0 & 0 \\
				\hline
				кот & скат & 2 & 2 & 2 & 2 \\
				\hline
				56 & two & 3 & 3 & 3 & 3 \\
				\hline
				exam & example & 3 & 3 & 3 & 3 \\
				\hline
				конфета & календула & 6 & 6 & 6 & 6 \\
				\hline
				abcde & edcba & 4 & 4 & 4 & 4 \\
				\hline
				dasha & arisah & 5 & 4 & 4 & 4 \\
				\hline
			\end{tabular}	
		\end{threeparttable}
	\end{center}
\end{table}


\section*{Вывод}

Были предоставлены листинги кода на выбранном языке программирования и приведена таблица с результатами выполнения программы на заданных тестовых данных.