\chapter*{Заключение}
\addcontentsline{toc}{chapter}{Заключение}

По окончании лабораторной работы, при изучении алгоритмов для вычисления расстояния Левенштейна и Дамерау - Левенштейна, успешно применили и усовершенствовали наши навыки в области динамического программирования и разработки программного обеспечения.

В результате исследования было определено, что время алгоритмов нахождения расстояний Левенштейна и Дамерау - Левенштейна растет в геометрической прогрессии при увеличении длин строк. 
Лучшие показатели по времени дает нерекурсивная реализация алгоритма нахождения расстояния Левенштейна за счет сохранения необходимых промежуточных вычислений. 
При этом итеративные реализации с использованием матрицы занимают больше память, чем рекурсивнаые реализации, поэтому в этом случае предпочтение отдается рекурсивным алгоритмам нахождения расстояния. 

В ходе выполнения лабораторной работы были выполнены следующие задачи: 
\begin{enumerate}[label={\arabic*)}]
	\item изучены расстояния Левенштейна и Дамерау - Левенштейна;
	\item разработаны и реализованы алгоритмы поиска расстояния Левенштейна и Дамерау - Левенштейна;
	\item создан программный продукт, позволяющий протестировать реализованные алгоритмы;
	\item проведен сравнительный анализ процессорного времени и затрачиваемой алгоритмами памяти;
	\item был подготовлен отчет по лабораторной работе.
\end{enumerate}