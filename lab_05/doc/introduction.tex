\chapter*{\hfill{\centering ВВЕДЕНИЕ}\hfill}
\addcontentsline{toc}{chapter}{ВВЕДЕНИЕ}

При обработке данных могут возникать ситуации, когда один набор данных необходимо обработать последовательно несколькими алгоритмами.
В таком случае удобно использовать конвейерную обработку данных, что позволяет на каждой следующей <<линии>> конвейера использовать данные, полученные с предыдущего этапа~\cite{conveyer-opr}.

Помимо линейной конвейерной обработки данных, существуют параллельная конвейерная обработка данных.
При таком подходе все линии работают с меньшим времени простоя по сравнению с линейным вариантом, так как могут обрабатывать задачи независимо от других линий.

Целью данной лабораторной работы является исследование параллельной и последовательной реализации конвейерной обработки данных.

В рамках выполнения работы необходимо решить следующие задачи:

\begin{itemize}
	\item описать конвейерную обработку данных;
	\item реализовать параллельную и линейную версию конвейерных вычислений;
	\item исследовать зависимость времени выполнения от количества задач для параллельной и линейной обработки конвейера;
	\item сделать выводы на основе проделанной работы.
\end{itemize}
