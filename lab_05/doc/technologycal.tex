\chapter{Технологическая часть}

В этом разделе предоставляются листинги реализованных алгоритмов и осуществляется выбор средств реализации.

\section{Средства реализации}

Для реализации программного обеспечения был выбран язык программирования С++~\cite{C++}.
Данный выбор обусловлен  наличием у языка встроенной библиотекой измерения процессорного времени и предоставлением возможности:
\begin{itemize}
	\item работа с потоками ядра предоставляется классом $thread$~\cite{thread};
	\item работа с очередями предоставляется классом $queue$~\cite{queue}.
\end{itemize}

\section{Реализация алгоритмов}

На листинге~\ref{lst:queue} представлены используемые типы очередей.

\begin{lstlisting}[label=lst:queue,caption=Реализация последовательной конвейерной обработки]
	std::queue<std::shared_ptr<Report>> q1;
	std::queue<std::shared_ptr<Report>> q2;
	std::queue<std::shared_ptr<Report>> q3;
\end{lstlisting}

\clearpage

На листинге~\ref{lst:lin} представлена последовательная реализация конвейерной обработки.

\begin{lstlisting}[label=lst:lin,caption=Реализация последовательной конвейерной обработки]
	void Conveyor::run_linear(size_t reports_cnt, bool log_fl) 
	{
		for (size_t i = 0; i < reports_cnt; i++)
		{
			std::string n = Conveyor::names[rand() % (names.size())];
			std::shared_ptr<Report> new_report(new Report(n));
			reports.push_back(new_report);
			q1.push(new_report);
		}
		for (size_t i = 0; i < reports_cnt; i++) 
		{
			std::shared_ptr<Report> report = q1.front();
			if (log_fl) report->get_text(i + 1);
			else report->get_text_meassure(i + 1);
			q2.push(report);
			q1.pop();
			report = q2.front();
			if (log_fl) report->find_subst(i + 1);
			else report->find_subst_meassure(i + 1);
			q3.push(report);
			q2.pop();
			report = q3.front();
			if (log_fl) report->output_res(i + 1);
			else report->output_res_meassure(i + 1);
			q3.pop();
		}
	}
\end{lstlisting}

\clearpage

На листинге~\ref{lst:par} представлена реализация главного потока при параллельной работе конвейера.

\begin{lstlisting}[label=lst:par,caption=Реализация основного потока для конвейерной обработки]
	void Conveyor::run_parallel(size_t reports_cnt, bool log_fl)
	{
		for (size_t i = 0; i < reports_cnt; i++)
		{
			std::string n = Conveyor::names[rand() % (names.size())];
			std::shared_ptr<Report> new_report(new Report(n));
			reports.push_back(new_report);
			q1.push(new_report);
		}
		
		if (log_fl)
		{
			this->threads[0] = std::thread(&Conveyor::get_text, this);
			this->threads[1] = std::thread(&Conveyor::find_subst, this);
			this->threads[2] = std::thread(&Conveyor::output_res, this);
		}
		else
		{
			this->threads[0] = std::thread(&Conveyor::get_text_meassure, this);
			this->threads[1] = std::thread(&Conveyor::find_subst_meassure, this);
			this->threads[2] = std::thread(&Conveyor::output_res_meassure, this);
		}
		
		for (int i = THRD_CNT-1; i >=0; i--)
		{
			this->threads[i].join();
		}
	}    
\end{lstlisting}

На листинге~\ref{lst:sh1} представлена реализация алгоритма извлечения текста из файла.
\begin{lstlisting}[label=lst:sh1,caption=Реализация вспомогательного потока, отвечающего за извлечения текста из файла]
	void Conveyor::get_text()
	{
		size_t task_num = 0;
		while (!this->q1.empty())
		{
			std::shared_ptr<Report> report = q1.front();
			q1.pop();
			report->get_text(++task_num);
			mutex_q2.lock();
			q2.push(report);
			mutex_q2.unlock();
		}
	}
\end{lstlisting}

На листинге~\ref{lst:sh2} представлена реализация алгоритма работы 2 рабочего потока, который выполняет поиск
вхождений подстроки в строку.
\begin{lstlisting}[label=lst:sh2,caption=Реализация вспомогательного потока, отвечающего за поиск подстроки в строке, language=C++]
	void Conveyor::find_subst()
	{
		size_t task_num = 0;
		do
		{
			if (!this->q2.empty())
			{
				mutex_q2.lock();
				std::shared_ptr<Report> report = q2.front();
				mutex_q2.unlock();
				report->find_subst(++task_num);
				mutex_q3.lock();
				q3.push(report);
				mutex_q3.unlock();
				mutex_q2.lock();
				q2.pop();
				mutex_q2.unlock();
			}
		} while(!q1.empty() || !q2.empty());
	}
\end{lstlisting}

На листинге~\ref{lst:sh3} представлена реализация алгоритма сохранения результатов поиска в файл.
\begin{lstlisting}[label=lst:sh3,caption=Реализация вспомогательного потока, отвечающего за сохранение результатов в файл, language=C++]
	void Conveyor::output_res()
	{
		size_t task_num = 0;
		do
		{
			if (!this->q3.empty())
			{
				mutex_q3.lock();
				std::shared_ptr<Report> report = q3.front();
				mutex_q3.unlock();
				report->output_res(++task_num);
				mutex_q3.lock();
				q3.pop();
				mutex_q3.unlock();
			}
		} while (!q1.empty() || !q2.empty() || !q3.empty());
	}   
\end{lstlisting}

\section*{Вывод}

В данном разделе была разработана реализация конвейерных вычислений.
Также в данном разделе была приведена информация о выбранных средствах для разработки алгоритмов.