\chapter{Аналитическая часть}

В данном разделе представлены теоретические сведения о рассматриваемых алгоритмах.

\section{Конвейерная обработка данных}

Конвейер --- система поточного производства.
В терминах программирования ленты конвейера представлены функциями, выполняющими над неким набором данных операции и предающие их на следующую ленту конвейера.
Моделирование конвейерной обработки хорошо сочетается с технологией многопоточного программирования --- под каждую ленту конвейера выделяется отдельный поток, все потоки работают в асинхронном режиме~\cite{conveyer}.

\section{Описание задачи}

В качестве примера для конвейерной обработки будет обрабатываться
текстовый файл.
Всего будет использовано три ленты, выполняющих следующие функции:
\begin{itemize}
	\item извлечение строки(текста) из файла;
	\item поиск всех вхождений подстроки в строку по алгоритму Бойера---Мура;
	\item сохранение результатов поиска в отдельный файл.
\end{itemize}

Ленты конвейера (обработчики) будут передавать друг другу заявки, каждая из которых будет содержать:
\begin{itemize}
	\item имя файла, содержащего строку;
	\item строку из файла;
	\item искомую подстроку;
	\item результаты поиска подстроки в строке.
\end{itemize}

\section*{Вывод}

В данном разделе были рассмотренны особенности построения конвейерных вычислений.
