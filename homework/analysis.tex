\chapter{Аналитическая часть}

\section{Графовые модели программы}

Программа представлена в виде графа -- набор вершин и множество соединяющих их направленных дуг.

\begin{itemize}
	\item Вершины -- процедуры, циклы, линейные участки, операторы, итерации циклов, срабатывание операторов и т.д.
	\item Дуги -- отражают связь (отношение между вершинамы).
\end{itemize}

Выделяют 2 типа отношений:

\begin{itemize}
	\item операционное отношение -- по передаче управления;
	\item информационное отношение -- по передаче данных.
\end{itemize}

Граф управления -- модель, в который вершины -- операторы, дуги -- операционные отношения.

Информационный граф -- модель, в которой вершины -- операторы, дуги~-- информационные отношения.

Операционная история -- модель, в которой вершины -- срабатывание
операторов, дуги -- операционные отношения.

Информационная история -- модель, в которой вершины -- срабатывание операторов, дуги -- информационные отношения.