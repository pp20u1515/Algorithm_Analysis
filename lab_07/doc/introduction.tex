\chapter*{\hfill{\centering ВВЕДЕНИЕ}\hfill}
\addcontentsline{toc}{chapter}{ВВЕДЕНИЕ}

В мире программирования и алгоритмов поиск элемента в массиве -- одна из фундаментальных задач, которая выступает в роли ключа к разгадке многих аспектов эффективной обработки данных. 
Эта задача, хоть и проста на первый взгляд, замаскирована внутри себя множеством сложных решений и стратегий. 
Каждый подход несет в себе не только методику нахождения элемента, но и философию взаимодействия с множеством данных.

Целью данной лабораторной работы является изучение алгоритмов поиска в массиве -- стандартный алгоритм и алгоритм бинарного поиска.

Для достижения цели, требуется выполнить следующие задачи:

\begin{enumerate}[label={\arabic*)}]
	\item проанализировать стандартный алгоритм и алгоритм бинарного поиска с единственным сравнением для каждого медианного элемента;
	\item определить средства программной реализации выбранного алгоритма;
	\item провести сравнительный анализ по времени реализованного алгоритма;
	\item подготовить отчет о лабораторной работе.
\end{enumerate}