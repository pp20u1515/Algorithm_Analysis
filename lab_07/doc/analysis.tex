\chapter{Аналитическая часть}
\section{Массив}

Массив -- тип данных, в котором хранится упорядоченный набор однотипных элементов. 

Поиск -- обработка некоторого множества данных с целью выявления подмножества данных, соответствующего критериям поиска~\cite{search}.

Все алгоритмы поиска делятся на:

\begin{itemize}
	\item поиск в неупорядоченном множестве данных;
	\item поиск в упорядоченном множестве данных.
\end{itemize}

Упорядоченность -- наличие отсортированного ключевого поля.

\section{Алгоритм бинарного поиска}

Бинарный поиск -- тип поискового алгоритма, который последовательно делит пополам заранее отсортированный массив данных, чтобы обнаружить нужный элемент~\cite{binary_search}.

\section{Стандартный алгоритм поиска}

Стандартный алгоритм поиска -- метод решения, при котором поочередно перебираются все элементы массива, пока не будет найден нужный.
Сложность такого алгоритма зависит от количества всех возможных решений, а время решения может стремиться к экспоненциальному времени работы. 
Чем дальше искомый ключ от начала массива, тем выше трудоемкость алгоритма. 

\section*{Вывод}

В данном разделе была представлена информация о алгоритмах поиска элемента в массиве.