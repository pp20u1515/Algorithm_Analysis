\chapter*{\hfill{\centering ВВЕДЕНИЕ}\hfill}
\addcontentsline{toc}{chapter}{ВВЕДЕНИЕ}

С появлением новых вычислительных систем программисты стали сталкиваться с необходимостью проведения одновременной обработки данных для улучшения отзывчивости системы, ускорения выполнения вычислений и более эффективного использования вычислительных ресурсов.
Развитие процессоров позволило использовать один процессор для выполнения нескольких параллельных операций, что привело к появлению термина «Многопоточность».

Целью данного рубежного контроля является исследование обычный и модифицированный алгоритм исправления орфографических ошибок в тексте (алгоритм поиска расстояния Левенштейна).

Для достижения цели, требуется выполнить следующие задачи:

\begin{enumerate}[label={\arabic*)}]
	\item проанализировать параллельный вариант алгоритма исправления орфографических ошибок в тексте;
	\item определить средства программной реализации выбранного алгоритма;
	\item реализовать алгоритм поиска расстояния Левенштейна и модифицированный алгоритм поиска расстояния Левенштейна;
	\item провести сравнительный анализ по времени реализованного алгоритма;
	\item подготовить отчет о рубежном контроле.
\end{enumerate}