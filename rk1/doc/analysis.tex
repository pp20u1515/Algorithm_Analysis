\chapter{Аналитическая часть}
\section{Многопоточность}

Многопоточность представляет собой способность центрального процессора (ЦП) одновременно обрабатывать несколько потоков, используя ресурсы одного процессора~\cite{multithreading}. 
Поток -- это последовательность инструкций, которые могут выполняться параллельно с другими потоками в рамках одного процесса, из которого они произошли.

Процесс представляет собой исполняемую программу во время ее выполнения~\cite{process}. При запуске программы создается процесс.
Процесс может включать в себя один или несколько потоков. 
Поток является частью процесса, ответственной за выполнение конкретных задач в рамках приложения. 
Завершение процесса происходит, когда все его потоки завершают свою работу. В операционной системе каждый поток представляет собой задачу, которую процессор должен выполнить. 
Современные процессоры могут обрабатывать несколько задач на одном ядре, создавая виртуальные ядра, или иметь несколько физических ядер, и такие процессоры называются многоядерными.

При разработке программы, которая использует несколько потоков, важно учитывать, что если потоки запускаются последовательно и управление передается каждому из них поочередно, то полный потенциал многозадачности не будет реализован. 
Это связано с тем, что выигрыш от параллельного выполнения задач не будет полностью использован. 
Эффективное использование многозадачности достигается путем создания потоков для независимых по данным задач и их параллельного выполнения, что позволяет сократить общее время выполнения процесса.

При работе с потоками возникает проблема общего доступа к данным. 
Одним из основных ограничений является запрет на одновременную запись в одну и ту же ячейку памяти из двух или более потоков.
Поэтому требуется использовать механизм синхронизации доступа к данным, который называется мьютексом. 
Этот мьютекс может быть захвачен одним потоком для монопольного доступа к данным или освобожден для доступа других потоков. Например, если два потока одновременно пытаются захватить мьютекс, только одному из них это удастся, а второй поток будет ожидать, пока мьютекс освободится.

Совокупность инструкций, которые выполняются между захватом и освобождением мьютекса, называется критической секцией.
Поскольку в период захвата мьютекса остальные потоки, которым требуется доступ к тем же данным для выполнения критической секции, ожидают освобождения мьютекса, необходимо стараться минимизировать объем операций в критической секции.

\section{Исправление орфографических ошибок в тексте}

Исправление орфографических ошибок в тексте представляет собой важный этап редактирования, направленный на улучшение грамматической корректности и читаемости текстовой информации~\cite{mistakes}.

Орфографические ошибки могут варьироваться от неверного написания отдельных слов до некорректного использования знаков препинания. Процесс исправления таких ошибок способствует повышению качества текста и обеспечивает более точное и понятное восприятие сообщаемой информации.

В данной лабораторной работе проводится распараллеливания алгоритма исправления орфографических ошибок в тексте. 
Для этого весь текст поровну рапределяется между всеми потоками.

В качестве одного из аргументов каждый поток получает выделенный
для него строку массива слов. 

\section*{Вывод}

В данном разделе была представлена информация о многопоточности
и исследуемом алгоритме.