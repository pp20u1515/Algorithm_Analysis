\chapter*{\hfill{\centering ЗАКЛЮЧЕНИЕ}\hfill}
\addcontentsline{toc}{chapter}{ЗАКЛЮЧЕНИЕ}

В результате исследования алгоритма поиска расстояния Левенштейна и его модификации обнаружили, что модифицированный вариант демонстрирует более высокую эффективность. 
Этот успех объясняется несколькими ключевыми оптимизациями.

Прежде всего, модификация алгоритма включает в себя использование одномерного массива вместо матрицы для хранения расстояний между символами. 
Такой подход существенно экономит память и сокращает количество операций обращения к ней, что в итоге приводит к повышению общей производительности.

Дополнительно, внедрена дополнительная переменная, которая заменяет значения двух ячеек (arr[i-1][j-1] и arr[i][j-1]). 
Это уменьшает необходимость постоянного обращения к ячейкам матрицы в рамках каждой итерации цикла, что также способствует улучшению общей производительности алгоритма.

Кроме того, применен оптимизированный подход к расчету замены. 
Расчет стоимости замены осуществляется прямо внутри формулы для определения минимального значения, что сокращает время, требуемое для вычисления изменения, и, следовательно, повышает эффективность алгоритма в целом.

Таким образом, модификация алгоритма поиска расстояния Левенштейна представляет собой успешное слияние оптимизированных методов, что приводит к более быстрой и эффективной обработке строк.

В результате выполнения рубежного контроля выполнены следующие задачи:

\begin{enumerate}[label={\arabic*)}]
	\item проанализирован параллельный вариант алгоритма исправления орфографических ошибок в тексте;
	\item определены средства программной реализации выбранного алгоритма;
	\item реализованы алгоритм поиска расстояния Левенштейна и модифицированный алгоритм поиска расстояния Левенштейна;
	\item проведен сравнительный анализ по времени реализованного алгоритма;
	\item подготовлен отчет о рубежном контроле.
\end{enumerate}