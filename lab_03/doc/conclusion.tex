\chapter*{Заключение}
\addcontentsline{toc}{chapter}{Заключение}

В результате исследования трудоемкости алгоритмов было установлено, что наименее затратным по времени в среднем случае оказался алгоритм сортировки бинарным деревом. 
Этот результат объясняется тем, что в среднем случае, когда массив заполняется произвольными значениями, бинарое дерево ближе к сбалансированному состоянию, что улучшает его производительность по сравнению с другими алгоритмами.
С другой стороны, в лучшем и худшем случае данный алгоритм проявляет неэффективность, поскольку находится в несбалансированном состоянии. 
В этих случаях предпочтение отдается алгоритму сортировки расческой, который в подобных сценариях продемонстрировал более высокую производительность.

Наименее затратным по памяти оказался алгоритм сортировки расчесткой из-за того, что он использует только структуру массива, пока остальные алгоритмы используют дополнительные структуры.
Таким образом, выбор алгоритма сортировки зависит от размера массива и характера данных, которые требуется упорядочить, а также от трудоемкости алгоритма.

В результате выполнения лабораторной работы были выполнены следующие задачи: 
\begin{enumerate}[label={\arabic*)}]
	\item разработаны и реализованы алгоритмы сортировки массивов;
	\item создан программный продукт, позволяющий протестировать реализованные алгоритмы;
	\item проведен сравнительный анализ процессорного времени выполнения реализаций данных алгоритмов;
	\item проведен сравнительный анализ затрачеваемой алгоритмами памяти;
	\item дана оценка трудоемкости алгоритмов;
	\item был подготовлен отчет по лабораторной работе.
\end{enumerate}