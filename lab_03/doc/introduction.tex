\chapter*{Введение}
\addcontentsline{toc}{chapter}{Введение}

В данной лабораторной работе будут расмотрены сортировки.
Сортировка -- преобразование последовательности элементов в неубывающую (или невозрастающую) последовательность\cite{definition}. 
В частности, одной из целей сортировки является облегчение последующего поиска элементов в отсортированном множестве.
Любой алгоритм сортировки можно разбить на три основные части:

\begin{itemize}
	\item сравнение элементов для определения их упорядоченности;
	\item перестановка элементов;
	\item сортирующий алгоритм, который осуществляет сравнение и перестановку элементов до тех пор, пока все элементы не будут упорядочены.
\end{itemize}

Самой важной характеристикой алгоритма сортировки является устойчивость.
Она заключается в сохранении относительного порядка объектов с одинаковыми значениями ключа.

Целью данной лабораторной работы является исследование трудоемкости алгоритмов сортировки. 
Для успешного выполнения лабораторной работы необходимо выполнить следующие задачи: 

\begin{enumerate}[label={\arabic*)}]
	\item изучить и реализовать 3 алгоритма сортировки: слиянием, бынарным деревом, расчесткой;
	\item создать программный продукт, позволяющий протестировать реализованные алгоритмы;
	\item выбрать инструменты для замера процессорного времени выполнения реализаций алгоритмов;
	\item провести анализ затрат работы алгоритмов по времени и по памяти;
	\item дать оценку трудоемкости алгоритмов;
	\item подготовить отчет по лабораторной работе.
\end{enumerate}