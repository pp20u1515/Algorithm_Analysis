\chapter{Аналитическая часть}

\section{Сортировка слиянием}

Алгоритм сортировки слиянием в большой степени соответствует парадигме метода разбиения\cite{merge_sort}. 
Сортировка слиянием применяется для структур данных, доступ к элементам которых можно получать только последовательно, например, списки или массивы.

Алгоритм сортировки для массива:

\begin{enumerate}[label={\arabic*)}]
	\item сортируемый массив разбивается на две части примерно одинакового размера;
	\item каждая из получившихся частей сортируется отдельно тем же самым алгоритмом;
	\item два упорядоченных массива половинного размера соединяются в один.
\end{enumerate}

Процесс разбиения задачи на более мелкие выполняется рекурсивно до тех пор, пока размер массива не уменьшится до одного элемента. 

\section{Сортировка бинарным деревом}

Из элементов массива формируется бинарное дерево поиска. 
Первый элемент -- корень дерева, остальные добавляются по следующему методу: начиная с корня дерева, элемент сравнивается с узлами. 
Если элемент меньше чем узел, тогда спускаемся по левой ветке, иначе по правой. 
Спустившись до конца, сам элемент становится узлом. 
Построенное таким образом дерево можно обойти так, чтобы двигаться от узлов с меньшими значениями к узлам с большими значениями. 
При этом получаем все элементы в возрастающем порядке\cite{bin_tree}.

\section{Сортировка расчесткой}

Основная идея «расчёстки» в том, чтобы первоначально брать достаточно большое расстояние между сравниваемыми элементами и по мере упорядочивания массива сужать это расстояние вплоть до минимального. 
Таким образом, мы как бы причёсываем массив, постепенно разглаживая на всё более аккуратные пряди. 

Для сравнения двух элементов рекомендуется учитывать специальную величину, известную как фактор уменьшения, оптимальное значение которой приблизительно равно 1,247. 
Сначала расстояние между элементами максимально, то есть равно размеру массива минус один. 
Затем, пройдя массив с этим шагом, необходимо поделить шаг на фактор уменьшения и пройти по списку вновь. 
Так продолжается до тех пор, пока разность индексов не достигнет единицы. 
В этом случае сравниваются соседние элементы как и в сортировке пузырьком, но такая итерация одна.
