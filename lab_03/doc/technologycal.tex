\chapter{Технологическая часть}

В этом разделе предоставляются листинги выполненных алгоритмов и осуществляется выбор средств реализации.

\section{Выбор средств реализации}

Для выполнения данной лабораторной работы был выбран язык программирования C++. 
Время измерялось с помощью функции \textit{clock()} из библиотеки \textit{time.h} \cite{cpp-lang-time}.

\section{Реализация алгоритмов}

В листингах 3.1 -- 3.3 представлены реализации рассматриваемых алгоритмов.

\hspace{0.6cm}В Листинге 3.1 показана реализация алгоритма сортировки слиянием.

\bigskip
\begin{lstlisting}[caption=Алгоритм сортировки слиянием]
	void merge_sort(vector<int> &array, size_t start, size_t end)
	{
		if (start < end)
		{
			size_t middle = start + (end - start) / 2;
			merge_sort(array, start, middle);
			merge_sort(array, middle + 1, end);
			merge(array, start, middle, end);
		}
	}
\end{lstlisting}
\clearpage

\hspace{0.6cm}В Листинге 3.2 показана реализация алгоритма сортировки бинарным деревом.
\begin{lstlisting}[caption=Алгоритм сортировки бинарным деревом]
	vector<int> bin_tree_sort(vector<int> &array)
	{
		node_t *root = nullptr;
		
		for (int data : array)
		root = insert(root, data);
		
		vector<int> result; 
		inorder_traversal(root, result);
		return result;   
	} 
\end{lstlisting}

В Листинге 3.3 показана реализация алгоритма сортировки расчесткой.
\begin{lstlisting}[caption=Алгоритм сортировки расчесткой]
	void coctail_sort(vector<int> &array)
	{
		double factor = 1.2473309;
		
		for (int step = array.size() - 1; step >= 1; step /= factor)
		    for (int index = 0; index + step < (int)array.size(); index++)
		        if (array[index] > array[index + step])
		            swap(array[index], array[index + step]);
	}
\end{lstlisting}

\section*{Вывод}

Были выбраны инструменты для реализации и разработаны алгоритмы упорядочивания объектов массива: сортировки слиянием, бинарным деревом и расчесткой. 
Также предоставлены листинги кода на выбранном языке программирования.