\chapter*{Заключение}
\addcontentsline{toc}{chapter}{Заключение}

В результате исследования трудоемкости алгоритмов был получен следующий вывод. Стандартный алгоритм умножения матриц занимает больше времени по сравнению с алгоритмами Винограда и Штрассена. 
Причина заключается в том, что в алгоритмах Винограда и Штрассена часть вычислений выполняется заранее. 
Это позволяет снизить сложность операций умножения. Сложность у этих алгоритмах $O(n^{2,37})$ и $O(n^{\log _{2}7})$, пока сложность у стандартного алгортма  $O(n^{3})$. 
Исходя из этого, алгоритм Винограда является более предпочтительным вариантом.

Однако лучшие результаты по времени показал оптимизированный алгоритм Винограда. 
Это достигается заменой операций и умножения на операции сдвига, что улучшает эффективность вычислений. 
Поэтому при выборе наиболее производительного алгоритма рекомендуется отдавать предпочтение оптимизированному алгоритму Винограда.

Несмотря на лучшую производительность по времени, алгоритм Винограда уступает в расходе памяти по сравнению со стандартным алгоритмом. 
Алгоритм Штрассена, в свою очередь, также требует дополнительной памяти для хранения промежуточных результатов. 
Это делает его менее эффективным, чем стандартный алгоритм и алгоритм Винограда. 
Это связано с тем, что в алгоритме Штрассена необходимо хранить дополнительные структуры данных, которые используются для оптимизации вычислений.

Таким образом, при выборе алгоритма умножения матриц, необходимо учитывать как производительность, так и расход памяти. 
Если необходимо сэкономить память, стандартный метод умножения матриц может оказаться более предпочтительным. 
Если нам важна скорость работы алгоритма, то оптимизированный алгоритм Винограда можеть оказаться более предпочтительным.

В результате выполнения лабораторной работы были выполнены следующие задачи: 
\begin{enumerate}[label={\arabic*)}]
	\item разработаны и реализованы алгоритмы умножения матриц;
	\item создан программный продукт, позволяющий протестировать реализованные алгоритмы;
	\item проведен сравнительный анализ процессорного времени выполнения реализаций данных алгоритмов;
	\item проведен сравнительный анализ затрачиваемой алгоритмами памяти;
	\item был подготовлен отчет по лабораторной работе.
\end{enumerate}