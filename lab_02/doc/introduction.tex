\chapter*{Введение}
\addcontentsline{toc}{chapter}{Введение}

В мире математики и информатики матрицы играют фундаментальную роль. Они представляют собой мощный инструмент для организации, хранения и обработки данных. Матрицы используются в различных областях, включая линейную алгебру, статистику, компьютерную графику, машинное обучение и многое другое. Они позволяют представить набор чисел в виде двумерной структуры, облегчая выполнение разнообразных математических операций.

Целью данной лабораторной работы является исследование алгоритмов умножения матриц. 
Для успешного выполнения лабораторной работы необходимо выполнить следующие задачи: 
\begin{enumerate}[label={\arabic*)}]
	\item изучить и реализовать 4 алгоритма умножения матриц: стандартный алгоритм умножения матриц, алгоритм Винограда, оптимизированый алгоритм Винограда и алгоритм Штрассена;
	\item создать программный продукт, позволяющий протестировать реализованные алгоритмы;
	\item выбрать инструменты для замера процессорного времени выполнения реализаций алгоритмов;
	\item провести анализ затрат работы алгоритмов по времени и по памяти;
	\item подготовить отчет по лабораторной работе.
\end{enumerate}