\chapter{Аналитическая часть}

Матрицей\cite{matrix} размера $m \times n$ называется прямоугольная таблица чисел, функций или алгебраических выражений, содержащая m строк и n столбцов. 
Числа m и n определяют размер матрицы. Числа, функции или алгебраические выражения, образующие матрицу, называются матричными элементами.
\begin{equation}
	\begin{pmatrix}
		a_{11} & a_{12} & \ldots & a_{1n}\\
		a_{21} & a_{22} & \ldots & a_{2n}\\
		\vdots & \vdots & \ddots & \vdots\\
		a_{m1} & a_{m2} & \ldots & a_{mn}
	\end{pmatrix},
\end{equation}

Основные операции на матрицами это:
\begin{enumerate}[label=\arabic*)]
	\item сложение матриц одинакового размера;
	\item вычитание матриц одинакового размера;
	\item умножение матриц в случае, если количество столбцов первой матрицы равно количеству строк второй матрицы. 
\end{enumerate}

\textit{Замечание:} операция умножения является некоммутативной. Это значит, что, оба произведения \text{AB} и {BA} двух квадратных матриц одинакового размера можно вычислить, но результаты будут отличаться друг от друга.

\section{Стандартный алгоритм}
Пусть даны две прямоугольные матрицы размерности $n \times m$ и $m \times p$
\begin{equation}
	A_{nm} = \begin{pmatrix}
		a_{11} & a_{12} & \ldots & a_{1m}\\
		a_{21} & a_{22} & \ldots & a_{2m}\\
		\vdots & \vdots & \ddots & \vdots\\
		a_{n1} & a_{n2} & \ldots & a_{nm}
	\end{pmatrix},
	\quad
	B_{mp} = \begin{pmatrix}
		b_{11} & b_{12} & \ldots & b_{1p}\\
		b_{21} & b_{22} & \ldots & b_{2p}\\
		\vdots & \vdots & \ddots & \vdots\\
		b_{m1} & b_{m2} & \ldots & b_{mp}
	\end{pmatrix},
\end{equation}

тогда матрица $C$
\begin{equation}
	C_{np} = \begin{pmatrix}
		c_{11} & c_{12} & \ldots & c_{1p}\\
		c_{21} & c_{22} & \ldots & c_{2p}\\
		\vdots & \vdots & \ddots & \vdots\\
		c_{n1} & c_{n2} & \ldots & c_{np}
	\end{pmatrix},
\end{equation}
где
\begin{equation}
	\label{eq:M}
	C_{ij} =
	\sum_{r=1}^{m} a_{ir}b_{rj} \quad (i=\overline{1,l}; j=\overline{1,n})
\end{equation}

будет называться произведением матриц $A$ и $B$.

\section{Алгоритм Винограда}

Алгоритм Винограда, предложенный Ш. Виноградом в 1987 году, представляет собой метод умножения квадратных матриц. Изначально алгоритм имел асимптотическую сложность $O(n^{2,3755})$, где $n$ - это размер стороны матрицы. 
Благодаря последующим усовершенствованиям, алгоритм Винограда теперь имеет лучшую асимптотику среди известных методов умножения матриц. 
Это делает его более эффективным и практичным.\cite{vinograd}.

Рассмотрим два вектора $V = (v_1, v_2, v_3, v_4)$ и $W = (w_1, w_2, w_3, w_4)$. Их скалярное произведение равно: $V \cdot W = v_1w_1 + v_2w_2 + v_3w_3 + v_4w_4$
Это равенство можно переписать в виде:
\begin{equation}
	\label{for:new}
	V \cdot W = (v_1 + w_2)(v_2 + w_1) + (v_3 + w_4)(v_4 + w_3) - v_1v_2 - v_3v_4 - w_1w_2 - w_3w_4.
\end{equation}

Второе выражение, несмотря на дополнительные операции (шесть умножений и десять сложений против четырех и трех), предоставляет преимущество в возможности предварительной обработки. 
Элементы этого выражения могут быть вычислены заранее и сохранены для каждой строки первой матрицы и каждого столбца второй матрицы. 
Это позволяет выполнять всего лишь два умножения и пять сложений для каждого элемента, при этом используя две заранее вычисленные суммы соседних элементов текущих строк и столбцов. 
Учитывая, что сложение более эффективно, чем умножение на современных вычислительных системах, этот алгоритм должен работать быстрее стандартного.

При нечетном размере матрицы $n$ требуется дополнительная операция: добавление произведения последних элементов строк и столбцов.

\section{Алгоритм Штрассена}

Идея алгоритма Штрассена: произведение матриц 2×2$(C=A\cdot B)$ вычисляется с использованием всего лишь семи умножений, вместо восьми, как в стандартном алгоритме\cite{book_strassen}. Действие осуществляется с использованием следующих формул: 

\begin{equation}
	\label{equ:shtassen}
	\begin{bmatrix}
		c_{00} & c_{01}\\
		c_{10} & c_{11}\\
	\end{bmatrix} = 
	\begin{bmatrix}
		a_{00} & a_{01}\\
		a_{10} & a_{11}\\
	\end{bmatrix} \cdot
	\begin{bmatrix}
		b_{00} & b_{01}\\
		b_{10} & b_{11}\\
	\end{bmatrix} = 
	\begin{bmatrix}
		m_{1} + m_{4} - m_{5} + m_{7} & m_{3} + m_{5}\\
		m_{2} + m_{4} & m_{1} + m_{3} - m_{2} + m_{6}\\
	\end{bmatrix}
\end{equation}
\begin{equation}
	m_{1} = (a_{00} + a_{11})(b_{00} + b_{11})
\end{equation}
\begin{equation}
	m_{2} = (a_{10} + a_{11}) \cdot b_{00}
\end{equation}
\begin{equation}
	m_{3} = (b_{01} - b_{11}) \cdot a_{00}
\end{equation}
\begin{equation}
	m_{4} = a_{11} \cdot (b_{10} - b_{00})
\end{equation}
\begin{equation}
	m_{5} = (a_{00} + a_{01}) \cdot b_{11}
\end{equation}
\begin{equation}
	m_{6} = (a_{10} - a_{00}) \cdot (b_{00} + b_{01})
\end{equation}
\begin{equation}
	m_{7} = (a_{01} - a_{11}) \cdot (b_{10} + b_{11})
\end{equation}

Метод Штрассена работает с квадратными матрицами, порядок которых можно представить в виде числа, равному степени двойки. 
В случае, когда это не так, матрица дополняется нулевыми элементами до квадратной матрицы ближайшей корректного размера.

\section*{Вывод}
В данном разделе были рассмотрены алгоритмы умножения матриц, а именно классический, алгоритм Штрассена и алгоритм Винограда. 
Также мы рассмотрели различные оптимизации, которые можно применить при программной реализации алгоритма Винограда.
