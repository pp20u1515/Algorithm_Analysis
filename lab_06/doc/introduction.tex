\chapter*{\hfill{\centering ВВЕДЕНИЕ}\hfill}
\addcontentsline{toc}{chapter}{ВВЕДЕНИЕ}

Задача коммивояжера --- задача комбинаторной оптимизации, цель которой --- нахождение самого выгодного маршрута, проходящего через указанные точки по одному разу с возвращением в исходную точку~\cite{tsp-about}.

Задача рассматривается как в симметричном, так и в ассиметричном варианте. В асимметричном варианте задача представляется в виде ориентированного графа, где веса рёбер между одними и теми же вершинами зависят от направления движения~\cite{tsp-about}.

Целью данной лабораторной работы является параметризация метода решения задачи коммивояжера на основе муравьиного метода.

Для достижения цели, требуется выполнить следующие задачи:

\begin{enumerate}[label={\arabic*)}]
	\item описать задачу коммивояжера;
	\item описать методы решения задачи коммивояжера: метод полного перебора и метод на основе муравьиного алгоритма;
	\item реализовать данные алгоритмы;
	\item провести сравнительный анализ по времени реализованного алгоритма;
	\item подготовить отчет о лабораторной работе.
\end{enumerate}