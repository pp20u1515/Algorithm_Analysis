\chapter{Аналитическая часть}
\section{Методы решения задачи коммивояжера}

\textbf{Полный перебор}.
Этот метод заключается в переборе всех возможных маршрутов в графе и выборе кратчайшего из них.
Сложность такого алгоритма~--- $O(n!)$~\cite{tsp-solutions}, где $n$~--- количество городов.

\textbf{Муравьиный алгоритм}.
Данный метод основан на принципах поведения колонии муравьев~\cite{tsp-solutions}.

Муравьи, двигаясь из муравейника в поисках пищи, откладывают феромоны на своем пути.
При этом, чем больше плотность феромона, тем короче путь, и, соответственно, чем длиннее путь, тем быстрее феромон испарится, и его плотность будет меньше~\cite{ants-alg}.

Со временем муравьи оставят наибольшее количество феромонов на самом коротком участке пути, что приведет к тому, что большинство муравьев выберет этот самый короткий путь, и, следовательно, они оставят еще больше феромонов на нем, что уменьшит вероятность выбора других маршрутов~\cite{ants-alg}.

При сведении алгоритма к математическим формулам, сначала определяется целевая функция:
\begin{equation}
	\label{eqn:target-func}
	n = \frac{1}{D},
\end{equation}
где $D$~--- расстояние до заданного пункта~\cite{ants-alg}.

Далее вычисляются вероятности перехода в заданную точку:
\begin{equation}
	P = \frac{t^e \cdot n^b}{\sum_{i=1}^{m} t_i^a \cdot n_i^b},
\end{equation}
где $a,b$~--- настраиваемые параметры,
\\ $t$~--- концентрация феромона~\cite{ants-alg}.

При $a = 0$ выбирается ближайший город и алгоритм становится <<жадным>> (выбирает только оптимальные или самые короткие расстояния), при $b = 0$ будут учитываться только след феромона, что может привести к сужению пространства поиска оптимального решения~\cite{ants-alg}. 

Последним производится обновление феромона:
\begin{equation}
	t_{i+1} = (1 - p) \cdot t_i + \frac{Q}{L_0},
\end{equation}
где $p$ настраивает скорость испарения феромона,
\\ $Q$ настраивает концентрацию нанесения феромона,
\\ $L_0$~--- длина пути на определенном участке~\cite{ants-alg}.

Последовательность вышеизложенных действий повторяется, пока не будет найден оптимальный маршрут.

Одной из модификаций муравьиного алгоритма является элитарная муравьиная система.
При таком подходе искусственно вводятся <<элитные>> муравьи, усиливающие уровень феромонов, оптимального на данный момент маршрута~\cite{elite-ants-alg}.


\section*{Вывод}

В данном разделе была рассмотрена задача коммивояжера, а также способы её решения: полным перебором и муравьиным алгоритмом.