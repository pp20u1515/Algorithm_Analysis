\chapter{Технологическая часть}

В этом разделе предоставляются листинги реализованных алгоритмов и
осуществляется выбор средств реализации.

\section{Выбор средств реализации}

Для выполнения данной лабораторной работы был выбран язык программирования C++. 
Время измерялось с помощью функции clock() из библиотеки time.h~\cite{clock_t}.

\section{Реализация алгоритмов}

В листинге 3.1 представлена реализация алгоритма полного перебора.

\begin{lstlisting}[caption=Алгоритм полного перебора]
	pair<int, vector<int>> brute_force_alg(const vector<vector<int>> &matrix, const unsigned int &amount_node) 
	{
		vector<int> permutation(amount_node);
		for (unsigned int i = 0; i < amount_node; i++)
			permutation[i] = i;
		
		int best_dist = INT_MAX;
		vector<int> best_route;
		
		do 
		{
			int cur_dist = 0;
			
			for (unsigned int i = 0; i < amount_node - 1; ++i)
				cur_dist += matrix[permutation[i]][permutation[i + 1]];
			
			if (cur_dist < best_dist) 
			{
				best_dist = cur_dist;
				best_route = permutation;
			}
		} while (next_permutation(permutation.begin(), permutation.end())); 
		
		return make_pair(best_dist, best_route);
	}
\end{lstlisting}

В листинге 3.2 представлена реализация алгоритма вычисления среднего значения расстояний между всеми парами.

\begin{lstlisting}[caption=Алгоритм вычисления среднего значения расстояний между всеми парами]
	double calcQ(const vector<vector<int>>& matrix) 
	{
		double sum = 0.0;
		int count = 0;
		
		for (const auto& row : matrix)
			for (int dist : row)
				if (dist != 0) 
					{
						sum += dist;
						count++;
					}
		return sum / count;
	}
\end{lstlisting}

\clearpage

В листинге 3.3 представлена реализация алгоритма выбора следующего города, куда муравей должен переместиться.

\begin{lstlisting}[caption=Алгоритм  выбора следующего города]
	int chooseNextLoc(const vector<double> &P) 
	{
		double posibility = static_cast<double>(rand()) / RAND_MAX;
		double cur_posibility = 0.0;
		int to = 0;
		
		while (cur_posibility < posibility && to < P.size()) 
		{
			cur_posibility += P[to];
			to++;
		}
		
		return to - 1;
	}  
\end{lstlisting}

В листинге 3.4 представлена реализация алгоритма вычисления пути, пройденного муравьем.

\begin{lstlisting}[caption=Алгоритм  вычисления пути]
	double calcLen(const vector<vector<int>> &matrix, const vector<int> &ant_route) 
	{
		double ant_dist = 0.0;
		
		for (size_t i = 0; i < ant_route.size() - 1; ++i) 
		{
			int city_from = ant_route[i];
			int city_to = ant_route[i + 1];
			ant_dist += matrix[city_from][city_to];
		}
		return ant_dist;
	}
\end{lstlisting}

\clearpage

В листинге 3.5 представлена реализация муравьиного алгоритма.

\begin{lstlisting}[caption=Муравьиный алгоритм]
	pair<double, vector<int>> fit(const vector<vector<int>> &matrix, const int &ants) 
	{
		srand(static_cast<unsigned int>(time(nullptr)));
		int cities = matrix.size();
		vector<vector<double>> tao(cities, vector<double>(cities, 1.0)); 
		
		vector<vector<int>> ant_routes(ants, vector<int>(cities, 0)); 
		vector<double> ant_dist(ants, 0.0); 
		
		vector<int> best_route; 
		double best_dist = numeric_limits<double>::infinity();
		
		Q = calcQ(matrix);
		
		for (int day = 0; day < days; ++day) 
		{
			for (int ant = 0; ant < ants; ++ant) 
			{
				ant_routes[ant][0] = rand() % cities;
				
				for (int i = 1; i < cities; ++i) 
				{
					int from_city = ant_routes[ant][i - 1];
					
					vector<double> P(cities, 0.0);
					double sum_prob = 0.0;
					
					for (int j = 0; j < cities; ++j) 
					{
						if (find(ant_routes[ant].begin(), ant_routes[ant].begin() + i, j) == ant_routes[ant].begin() + i) 
						{
							double pher = pow(tao[from_city][j], a);
							double dist = pow(1.0 / matrix[from_city][j], b);
							P[j] = pher * dist;
							sum_prob += P[j];
						}
					}
					
					for (int j = 0; j < cities; ++j)
					P[j] /= sum_prob;
					
					int next_city = chooseNextLoc(P);
					ant_routes[ant][i] = next_city;
				}
				
				ant_dist[ant] = calcLen(matrix, ant_routes[ant]);
				
				if (ant_dist[ant] < best_dist) 
				{
					best_dist = ant_dist[ant];
					best_route = ant_routes[ant];
				}
			}
			for (int ant = 0; ant < ants; ++ant) 
			{
				double delta_tao = Q / ant_dist[ant];
				for (int i = 0; i < cities - 1; ++i) 
				{
					int city_from = ant_routes[ant][i];
					int city_to = ant_routes[ant][i + 1];
					tao[city_from][city_to] += delta_tao;
					
					if (tao[city_from][city_to] < MIN_PHER) {
						tao[city_from][city_to] = MIN_PHER;
					}
				}
			}
			for (int i = 0; i < cities; ++i)
				for (int j = 0; j < cities; ++j)
					tao[i][j] *= (1.0 - rho);
		}
		return make_pair(best_dist, best_route);
	}
\end{lstlisting}

\section*{Вывод}

Были представлены листинги всех реализаций алгоритмов --- полного перебора и муравьиного. 
Также в данном разделе была приведена информация о выбранных средствах для разработки алгоритмов.