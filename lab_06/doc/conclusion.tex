\chapter*{\hfill{\centering ЗАКЛЮЧЕНИЕ}\hfill}
\addcontentsline{toc}{chapter}{ЗАКЛЮЧЕНИЕ}

В результате исследования было получено, что использование муравьиного алгоритма наиболее эффективно при больших размерах матриц. Так, при размере матрицы, меньшим 9, муравьиный алгоритм медленнее алгоритма полного перебора, а при размере матрицы, равном 9 и  выше, муравьиный алгоритм быстрее алгоритма полного перебора. 
Следовательно, при размерах матриц больше 9 следует использовать муравьиный алгоритм, но стоит учитывать, что он не гарантирует получения глобального оптимума при решении задачи.

В результате выполнения лабораторной работы были выполнены следующие задачи:

\begin{enumerate}[label={\arabic*)}]
	\item описана задача коммивояжера;
	\item описаны методы решения задачи коммивояжера: метод полного перебора и метод на основе муравьиного алгоритма;
	\item реализованы данные алгоритмы;
	\item проведен сравнительный анализ по времени реализованного алгоритма;
	\item подготовлен отчет о лабораторной работе.
\end{enumerate}